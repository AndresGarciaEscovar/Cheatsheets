%%%%%%%%%%%%%%%%%%%%%%%%%%%%%%%%%%%%%%%%%%%%%%%%%%%%%%%%%%%%%%%%%%%%%%%%%%%%%%%%%%%%%%%%%%%%%%%%%%%%
% Snippets" Cheatsheet
%%%%%%%%%%%%%%%%%%%%%%%%%%%%%%%%%%%%%%%%%%%%%%%%%%%%%%%%%%%%%%%%%%%%%%%%%%%%%%%%%%%%%%%%%%%%%%%%%%%%
\section{\textquotedblleft Snippets\textquotedblright\ Cheatsheet}

%///////////////////////////////////////////////////////////////////////////////////////////////////
%  Conda - .bashrc conda Initialization
%///////////////////////////////////////////////////////////////////////////////////////////////////

\subsection{Conda - .bashrc conda Initialization}

The following code initializes the \textquotedblleft conda\textquotedblright\ environment in the
\textquotedblleft .bashrc\textquotedblright\ file. It also deactivates the \textquotedblleft 
conda\textquotedblright\ environment.
\begin{minted}[frame=single]{bash}
# >>> conda initialize >>>
_conda_setup="$('<path_to_environment>/bin/conda' 'shell.bash' 'hook' 2> /dev/null)"
if [ $? -eq 0 ]; then
    eval "$_conda_setup"
else
    if [ -f "<path_to_environment>/etc/profile.d/conda.sh" ]; then
        . "<path_to_environment>/etc/profile.d/conda.sh"
    else
        export PATH="<path_to_environment>/bin:$PATH"
    fi
fi
unset _conda_setup
# <<< conda initialize <<<

# Deactivate conda.
conda deactivate
\end{minted}

example:
\begin{minted}[frame=single]{bash}
# >>> conda initialize >>>
_conda_setup="$('/home/hp/miniconda3/bin/conda' 'shell.bash' 'hook' 2> /dev/null)"
if [ $? -eq 0 ]; then
    eval "$_conda_setup"
else
    if [ -f "/home/hp/miniconda3/etc/profile.d/conda.sh" ]; then
        . "/home/hp/miniconda3/etc/profile.d/conda.sh"
    else
        export PATH="/home/hp/miniconda3/bin:$PATH"
    fi
fi
unset _conda_setup
# <<< conda initialize <<<

# Deactivate conda.
conda deactivate
\end{minted}


%///////////////////////////////////////////////////////////////////////////////////////////////////
%  Python Code - Code Timer.
%///////////////////////////////////////////////////////////////////////////////////////////////////

\newpage
\subsection{Python - Live Elapsed Time}

For a subprocess:
\begin{minted}[frame=single]{python3}
"""
    Code to run a live timer for a subprocess.
"""

# ------------------------------------------------------------------------------------
# Imports
# ------------------------------------------------------------------------------------

# General
import subprocess

from datetime import datetime

# ------------------------------------------------------------------------------------
# Variables
# ------------------------------------------------------------------------------------

# Time variables.
time_ela = None  # Elapsed time.
time_str = None  # Time at which the process starts.

# Other
command = ["sh", "script.sh"]  # Command to run, this is an example.
elapsed = None  # Elapsed time counter.
flag_process = True  # Flag to indicate if the process is running.
process = None  # Contains the thread of the process.

# ------------------------------------------------------------------------------------
# Algorithm
# ------------------------------------------------------------------------------------

# Initial time is always 00:00
print(f"\rElapsed time: {int(0):02d}:{int(0):02d}", end="")

# Start infinite loop.
while True:
    # Only if the process is running.
    if not flag:
        elapsed = (datetime.now() - time_str).total_seconds()
        minutes = elapsed / 60
        seconds = (minutes - int(minutes)) * 60
        print(f"\rElapsed time: {int(minutes):02d}:{int(seconds):02d}", end="")

\end{minted}
\newpage
\begin{minted}[frame=single]{python3}

    # Only spawn one process.
    if flag:
        # Set the flag to False, start timer and process.
        flag = False
        str_time = datetime.now()
        proc = subprocess.Popen(
            command, stdout=subprocess.PIPE, stderr=subprocess.STDOUT
        )
        continue
    
    # Exit the loop if the process is done.
    if proc.returncode is not None:
        fstring = f"\rElapsed time: {int(minutes):02d}:{int(seconds):02d}"
        break

# Print the final time.
print(f"\r{fstring}")

\end{minted}
\subsection*{Things to Consider}
\begin{itemize}
    \item Remember to include a carriage return so the string is always removed.
    \item Include the carriage return at the \textbf{\textit{start}} of the string,
          \textbf{\textit{not}} at the end.
\end{itemize}
