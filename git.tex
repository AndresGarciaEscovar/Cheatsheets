%%%%%%%%%%%%%%%%%%%%%%%%%%%%%%%%%%%%%%%%%%%%%%%%%%%%%%%%%%%%%%%%%%%%%%%%%%%%%%%%%%%%%%%%%%%%%%%%%%%%
%  "git" Cheatsheet
%%%%%%%%%%%%%%%%%%%%%%%%%%%%%%%%%%%%%%%%%%%%%%%%%%%%%%%%%%%%%%%%%%%%%%%%%%%%%%%%%%%%%%%%%%%%%%%%%%%%
\section{\textquotedblleft git\textquotedblright\ Cheatsheet}


%///////////////////////////////////////////////////////////////////////////////////////////////////
%  "git" - Cloning
%///////////////////////////////////////////////////////////////////////////////////////////////////
\subsection{Cloning}

\begin{itemize}
    %
    \item Clone a repository:
    \begin{minted}[frame=single]{bash}
git clone <url>  
    \end{minted}
    For example:
    \begin{minted}[frame=single]{bash}
git clone git@github.com:Python-World/example.git
    \end{minted}
    %
    \item Clone a repository from a specific branch:
    \begin{minted}[frame=single]{bash}
git clone -b <branch> --single-branch <url>  
    \end{minted}
    For example:
    \begin{minted}[frame=single]{bash}
git clone -b demo --single-branch git@github.com:Python-World/example.git
    \end{minted}
        %
    \item Clone a repository with submodules:
    \begin{minted}[frame=single]{bash}
git clone --recurse-submodules <url>  
    \end{minted}
    For example:
    \begin{minted}[frame=single]{bash}
git clone --recurse-submodules git@github.com:Python-World/example.git
    \end{minted}
\end{itemize}


%///////////////////////////////////////////////////////////////////////////////////////////////////
%  "git" - Deleting
%///////////////////////////////////////////////////////////////////////////////////////////////////
\subsection{Deleting}

\begin{itemize}
    %
    \item Delete a local branch:
    \begin{minted}[frame=single]{bash}
git branch -d <branch>
    \end{minted}
    For example:
    \begin{minted}[frame=single]{bash}
git branch -d demo
    \end{minted}
    %
    \item Delete a remote branch:
    \begin{minted}[frame=single]{bash}
git push origin --delete <branch> 
    \end{minted}
    For example:
    \begin{minted}[frame=single]{bash}
git push origin --delete demo
    \end{minted}
\end{itemize}


%///////////////////////////////////////////////////////////////////////////////////////////////////
%  "git" - Updating
%///////////////////////////////////////////////////////////////////////////////////////////////////
\subsection{Updating}

\begin{itemize}
    %
    \item Update the list of remote branches:
    \begin{minted}[frame=single]{bash}
git remote update origin --prune
    \end{minted}
    %
    \item Update branches with conflicts:
    \begin{enumerate}
        \item Checkout the branch to be updated, call it \mintinline{bash}|<branch_1>|:
        \begin{minted}[frame=single]{bash}
git checkout <branch_1>
        \end{minted}
        \item Update the local branch, i.e., update \mintinline{bash}|<branch_1>|:
        \begin{minted}[frame=single]{bash}
git fetch origin
        \end{minted}
        \item Merge merge \mintinline{bash}|<branch_2>| into \mintinline{bash}|<branch_1>|:
        \begin{minted}[frame=single]{bash}
git merge origin/<branch_2>  # branch 2 -> branch 1
        \end{minted}
        \item Solve conflicts.:
        \begin{minted}[frame=single]{bash}
git mergetool  # Select the changes to keep.
        \end{minted}
        \item Commit the changes:
        \begin{minted}[frame=single]{bash}
git commit -m "Solved conflicts."
        \end{minted}
    \end{enumerate}
\end{itemize}


%///////////////////////////////////////////////////////////////////////////////////////////////////
%  "git" - Submodules
%///////////////////////////////////////////////////////////////////////////////////////////////////
\subsection{Submodules}

\begin{itemize}
    %
    \item Add a submodule:
    \begin{minted}[frame=single]{bash}
git submodule add <url>
    \end{minted}
    For example:
    \begin{minted}[frame=single]{bash}
git submodule add git@github.com:Module-Folder/example.git
    \end{minted}
    %
    \item Add a submodule with a custom name:
    \begin{minted}[frame=single]{bash}
git submodule add --name <name> <url> <path>
    \end{minted}
    For example, we want to add a module named \mintinline{bash}|example|, in the
    \mintinline{bash}|tests| directory, that is located in the current folder:
    \begin{minted}[frame=single]{bash}
git submodule add tests git@github.com:Python-World/tests-example.git ./tests
    \end{minted}
    %
    \item Update submodules:
    \begin{minted}[frame=single]{bash}
git submodule update --init --recursive
    \end{minted}
    %
    \item Remove a submodule:
    \begin{enumerate}
        \item Delete the relevant section from the \mintinline{bash}|.gitmodules| file.
        \item Delete the relevant section from \mintinline{bash}|.git/config| directory.
        \item Run \mintinline{bash}|git rm --cached <path_to_submodule>| (no trailing slash).
        \item Commit and delete the now untracked submodule files.
    \end{enumerate}
\end{itemize}


%///////////////////////////////////////////////////////////////////////////////////////////////////
%  "git" - Tracking
%///////////////////////////////////////////////////////////////////////////////////////////////////
\subsection{Tracking}

\begin{itemize}
    %
    \item Create a local branch:
    \begin{minted}[frame=single]{bash}
git checkout <local_branch>
    \end{minted}
    add the \mintinline{bash}|-b| flag to switch to the branch inmediately after creating it:
    \begin{minted}[frame=single]{bash}
git checkout -b <local_branch>
    \end{minted}
    %
    \item Branch from target branch to another branch:
    \begin{minted}[frame=single]{bash}
git checkout <new_branch> <target_branch>
    \end{minted}
    add the \mintinline{bash}|-b| flag to switch to the branch inmediately after creating it:
    \begin{minted}[frame=single]{bash}
git checkout -b <new_branch> <target_branch>
    \end{minted}
    %
    \item Track a remote branch:
    \begin{minted}[frame=single]{bash}
git branch -u origin/<branch>
    \end{minted}
\end{itemize}
