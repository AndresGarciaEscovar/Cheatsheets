%%%%%%%%%%%%%%%%%%%%%%%%%%%%%%%%%%%%%%%%%%%%%%%%%%%%%%%%%%%%%%%%%%%%%%%%%%%%%%%%%%%%%%%%%%%%%%%%%%%%
% "Installations and Setup"
%%%%%%%%%%%%%%%%%%%%%%%%%%%%%%%%%%%%%%%%%%%%%%%%%%%%%%%%%%%%%%%%%%%%%%%%%%%%%%%%%%%%%%%%%%%%%%%%%%%%
\section{\textquotedblleft Installations and Setup\textquotedblright\
Cheatsheet
}

%///////////////////////////////////////////////////////////////////////////////////////////////////
% Linux - NeoVIM
%///////////////////////////////////////////////////////////////////////////////////////////////////

\subsection{Linux - NeoVIM}

% Installation
\subsubsection{Installation}

\begin{enumerate}
    %
    \item To install \textbf{NeoVIM} there are many methods. The suggested method is to use the
    \textit{snap} store, or equivalent, by using the command:
    \begin{mintedbox}{bash}
$ sudo snap install nvim --classic
    \end{mintedbox}
    This will install \textbf{NeoVIM} in the system.
    %
    \item The \textit{--classic} flag is used to install the software with full access to the
    system, which is required for some plugins and, most of the time, for the software to even
    install.
    %
    \item Make sure that the
    \textit{.bashrc} has the snap store path in the \nminted{text}{PATH} variable at the start of
    the list:
    \begin{mintedbox}{bash}
export PATH=/snap/bin:$PATH
    \end{mintedbox}
    %
    \item To make sure the installation was successful, run the command:
    \begin{mintedbox}{bash}
$ nvim --version
    \end{mintedbox}
    Make sure that this is not a \quoted{developer version}; i.e., the latest stable version.
    %
\end{enumerate}

\subsubsection{Setup - Lazy: Package Manager}
This procedure is to perform the basic setup for Lazy and get \textbf{NeoVIM} running in a basic
way.
\begin{enumerate}
    \item Make sure that \textbf{NeoVIM} is properly installed.
    \item If the directory \nminted{text}{~/.config/nvim} does not exist,
    create it:
    \begin{mintedbox}{bash}
$ mkdir -p ~/.config/nvim
    \end{mintedbox}
    %
    \item Create the file \nminted{text}{~/.config/nvim/init.lua} file in the directory:
    \begin{mintedbox}{bash}
touch ~/.config/nvim/init.lua
    \end{mintedbox}
    This is done for the modern version of \textbf{NeoVIM}, which uses the \textit{lua} language for
    configuration.
    %
    \item Create the \nminted{text}{~/.config/nvim/lua} directory, if
    it does not exist:
    \begin{mintedbox}{bash}
mkdir ~/.config/nvim/lua
    \end{mintedbox}
    %
    \item In the \nminted{text}{~/.config/nvim/lua} directory, create the three directories:
    \begin{itemize}
        \item config
        \item plugins
        \item util
    \end{itemize}
    This is where the different configuration files will be stored. Since this is the default place
    where \textbf{NeoVim} will look for the different files and directories, it is a good idea to
    keep the default structure.
    %
    \item At this point, the file and directory structure should look like:
    \begin{mintedbox}{bash}
<root == ~/.config/nvim>
| init.lua
|_ lua
    |_ config
    |_ plugins
    |_ util
    \end{mintedbox}
    %
    \item To dowload the packages and give them the basic configuration, create an
    \nminted{bash}{init.lua} in the \nminted{bash}{config} directory:
    \begin{mintedbox}{bash}
<root == ~/.config/nvim>
| init.lua
|_ lua
    |_ config
    |   |_ init.lua
    |_ plugins
    |_ util
    \end{mintedbox}
    %
    \item Open the file \nminted{bash}{<root>/lua/config/init.lua} file and add the following
    lines:
    \begin{mintedbox}{lua}
local lazypath = vim.fn.stdpath("data") .. "/lazy/lazy.nvim"
if not vim.loop.fs_stat(lazypath) then
  vim.fn.system({
    "git",
    "clone",
    "--filter=blob:none",
    "https://github.com/folke/lazy.nvim.git",
    "--branch=stable", -- latest stable release
    lazypath,
  })
end
vim.opt.rtp:prepend(lazypath)

-- Example using a list of specs with the default options
-- Make sure to set `mapleader` before lazy so your mappings are correct
vim.g.mapleader = " "

require("lazy").setup({
  "folke/which-key.nvim",
  { "folke/neoconf.nvim", cmd = "Neoconf" },
  "folke/neodev.nvim",
})
    \end{mintedbox}
    %
    \item Use \textbf{NeoVIM} to open the file and, in \textbf{NeoVIM} command
    console mode (press \keystroke{esc}), run the command:
    command:
    \begin{mintedbox}{text}
~
-------------------------------------
:checkhealth
    \end{mintedbox}
\end{enumerate}

%///////////////////////////////////////////////////////////////////////////////////////////////////
% Linux - TeXLive
%///////////////////////////////////////////////////////////////////////////////////////////////////

\subsection{Linux - TeXLive}

These are the instructions to install and set up \textbf{TeXLive} in a Linux machine. The
installation instructions can be found in:
\begin{center}
    \url{https://www.tug.org/texlive/quickinstall.html}
\end{center}
\begin{enumerate}
    \item Go to the directory where you want to download the package:
    \begin{mintedbox}{bash}
cd /path/to/directory
    \end{mintedbox}
    %
    \item Navigate to the aforementioned web page and go to the \quoted{\textbf{tl;dr; Unix(ish)}}
    section to follow the instructions; repeated here for convenience.
    %
    \item Donwload the latest TeXLive package from one of the two sources:
    \begin{itemize}
        \item Use the \nminted{bash}{wget} command to download the package:
        \begin{mintedbox}{bash}
wget https://mirror.ctan.org/systems/texlive/tlnet/install-tl-unx.tar.gz
        \end{mintedbox}
        \item Use the \nminted{bash}{curl} command to download the package:
        \begin{mintedbox}{bash}
curl -L -o install-tl-unx.tar.gz \
https://mirror.ctan.org/systems/texlive/tlnet/install-tl-unx.tar.gz
        \end{mintedbox}
    \end{itemize}
\end{enumerate}

%///////////////////////////////////////////////////////////////////////////////////////////////////
% Miscellaneous - Setup SSH Keys to Connect Between Machines
%///////////////////////////////////////////////////////////////////////////////////////////////////

\subsection{Miscellaneous - Setup SSH Keys to Connect Between Machines}

The instructions are for setting up SSH keys to connect between machines, it should be useful for
any Linux distribution, MacOS, and Windows.\bigbreak

For this, we will need \textbf{OpenSSH} installed in both the origin and host machines. You can look
for the installation instructions for your specific system.

% Installation
\subsubsection{Installation}

We want to connect from \textit{machine$\_0$} to \textit{machine$\_1$} without having to type
the password every time.\bigbreak

\underline{\underline{Setup for machine$\_0$}}:

\begin{itemize}
    \item Create and ssh key; this key will have a private and public part. The public part will be
    copied to the receiving machine, i.e., \textit{machine$\_1$}. The private key will be kept in
    the sending machine, i.e., \textit{machine$\_0$}:
    \begin{mintedbox}{bash}
ssh-keygen -t rsa -b 4096
    \end{mintedbox}
    This will create a key with 4096 bits of encryption (suggested).
    \begin{itemize}
        \item To add a comment at the end of the key, use the \nminted{text}{-C} flag:
        \begin{mintedbox}{bash}
ssh-keygen -t rsa -b 4096 -C "Comment here"
        \end{mintedbox}
    \end{itemize}
    \item The command will ask for a location to save the key, make sure to choose the default
    location, which is \nminted{text}{~/.ssh} directory; the name you can choose:
    \begin{mintedbox}{text}
Enter file in which to save the key (/home/user/.ssh/id_rsa): /home/user/.ssh/name
    \end{mintedbox}
    \item If you want to add a passphrase to the key, you can do so. This will add an extra layer of
    security to the key. If you do not want to add a passphrase, just press \keystroke{enter}:
    \begin{mintedbox}{text}
Enter passphrase (empty for no passphrase):
    \end{mintedbox}
    \item The key will be created and saved in the location you specified. There will be two files:
    \nminted{text}{name} and \nminted{text}{name.pub}, where \nminted{text}{name} is the private key
    and \nminted{text}{name.pub} is the public key.
    \item Copy the public key such that it can be added to the receiving
    machine.
    \item To make the connection automatic, you can use the file \nminted{text}{~/.ssh/config} to
    setup the connection. This file should look like:
    \begin{mintedbox}{text}
Host machine_1
    HostName machine_1
    User user
    IdentityFile ~/.ssh/name
    \end{mintedbox}
\end{itemize}

\underline{\underline{Setup for machine$\_1$}}:

\begin{itemize}
    \item With the public key in hand, copy the contents of the file into the
    \nminted{text}{~/.ssh/authorized_keys} file in the receiving machine:
    \begin{mintedbox}{bash}
cat name.pub >> ~/.ssh/authorized_keys
    \end{mintedbox}
    \begin{itemize}
        \item For Windows, in Powershell (admin mode), use the following:
        \begin{mintedbox}{bash}
Get-Content name.pub >> ~/.ssh/authorized_keys
        \end{mintedbox}
    \end{itemize}
    \item Make sure that the permissions for the \nminted{text}{~/.ssh} directory are set to
    \nminted{text}{700} and the permissions for the \nminted{text}{~/.ssh/authorized_keys} file are
    set to \nminted{text}{600}:
    \begin{mintedbox}{bash}
chmod 700 ~/.ssh
chmod 600 ~/.ssh/authorized_keys
    \end{mintedbox}
    For Windows there is no need to change the permissions.
\end{itemize}
