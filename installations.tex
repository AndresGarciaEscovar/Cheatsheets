%%%%%%%%%%%%%%%%%%%%%%%%%%%%%%%%%%%%%%%%%%%%%%%%%%%%%%%%%%%%%%%%%%%%%%%%%%%%%%%%
% "Installations and Setup"
%%%%%%%%%%%%%%%%%%%%%%%%%%%%%%%%%%%%%%%%%%%%%%%%%%%%%%%%%%%%%%%%%%%%%%%%%%%%%%%%
\section{\textquotedblleft Installations and Setup\textquotedblright\
Cheatsheet
}

%///////////////////////////////////////////////////////////////////////////////
% Linux - NeoVIM 
%///////////////////////////////////////////////////////////////////////////////

\subsection{Linux - NeoVIM}

% Installation
\subsubsection{Installation}

\begin{enumerate}
    %
    \item To install \textbf{NeoVIM} there are many methods. The suggested
    method is to use the \textit{snap} store, or equivalent, by using the
    command:
    \begin{minted}[frame=single]{bash}
$ sudo snap install nvim --classic
    \end{minted}
    This will install \textbf{NeoVIM} in the system.
    %
    \item The \textit{--classic} flag is used to install the software with
    full access to the system, which is required for some plugins and, most of
    the time, for the software to even install.
    %
    \item Make sure that the
    \textit{.bashrc} has the snap store path in the \mintinline{text}|PATH|
    variable at the start of the list:
    \begin{minted}[frame=single]{bash}
export PATH=/snap/bin:$PATH
    \end{minted}
    %
    \item To make sure the installation was successful, run the command:
    \begin{minted}[frame=single]{bash}
$ nvim --version
    \end{minted}
    Make sure that this is not a \textquotedblleft developer 
    version\textquotedblright; i.e., the latest stable version.
    %
\end{enumerate}

\subsubsection{Setup - Lazy: Package Manager}
This procedure is to perform the basic setup for Lazy and get \textbf{NeoVIM}
running in a basic way.
\begin{enumerate}
    \item Make sure that \textbf{NeoVIM} is properly installed.
    \item If the directory \mintinline{text}|~/.config/nvim| does not exist,
    create it:
    \begin{minted}[frame=single]{bash}
$ mkdir -p ~/.config/nvim
    \end{minted}
    %
    \item Create the file \mintinline{text}|~/.config/nvim/init.lua| file in the
    directory:
    \begin{minted}[frame=single]{bash}
touch ~/.config/nvim/init.lua
    \end{minted}
    This is done for the modern version of \textbf{NeoVIM}, which uses the
    \textit{lua} language for configuration.
    %
    \item Create the \mintinline{text}|~/.config/nvim/lua| directory, if 
    it does not exist:
    \begin{minted}[frame=single]{bash}
mkdir ~/.config/nvim/lua
    \end{minted}
    %
    \item In the \mintinline{text}|~/.config/nvim/lua| directory, create the
    three directories:
    \begin{itemize}
        \item config
        \item plugins
        \item util
    \end{itemize}
    This is where the different configuration files will be stored. Since 
    this is the default place where \textbf{NeoVim} will look for the different
    files and directories, it is a good idea to keep the default structure.
    %
    \item At this point, the file and directory structure should look like:
    \begin{minted}[frame=single]{bash}
<root == ~/.config/nvim>
| init.lua
|_ lua
    |_ config
    |_ plugins
    |_ util
    \end{minted}
    %
    \item To dowload the packages and give them the basic configuration, create
    an \mintinline{bash}|init.lua| in the \mintinline{bash}|config| directory: 
    \begin{minted}[frame=single]{bash}
<root == ~/.config/nvim>
| init.lua
|_ lua
    |_ config
    |   |_ init.lua    
    |_ plugins
    |_ util
    \end{minted}
    %
    \item Open the file \mintinline{bash}|<root>/lua/config/init.lua| file and
    add the following lines:
    \begin{minted}[frame=single]{lua}
local lazypath = vim.fn.stdpath("data") .. "/lazy/lazy.nvim"
if not vim.loop.fs_stat(lazypath) then
  vim.fn.system({
    "git",
    "clone",
    "--filter=blob:none",
    "https://github.com/folke/lazy.nvim.git",
    "--branch=stable", -- latest stable release
    lazypath,
  })
end
vim.opt.rtp:prepend(lazypath)

-- Example using a list of specs with the default options
-- Make sure to set `mapleader` before lazy so your mappings are correct
vim.g.mapleader = " " 

require("lazy").setup({
  "folke/which-key.nvim",
  { "folke/neoconf.nvim", cmd = "Neoconf" },
  "folke/neodev.nvim",
})
    \end{minted}
    %
    \item Use \textbf{NeoVIM} to open the file and, in \textbf{NeoVIM} command
    console mode (press \keystroke{esc}), run the command:
    command:
    \begin{minted}[frame=single]{text}
~
-------------------------------------
:checkhealth
    \end{minted}
\end{enumerate}

%///////////////////////////////////////////////////////////////////////////////
% Miscellaneous - Setup SSH Keys to Connect Between Machines 
%///////////////////////////////////////////////////////////////////////////////

\subsection{Miscellaneous - Setup SSH Keys to Connect Between Machines}

The instructions are for setting up SSH keys to connect between machines, it 
should be useful for any Linux distribution, MacOS, and Windows.\bigbreak

For this, we will need \textbf{OpenSSH} installed in both the origin and host
machines. You can look for the installation instructions for your specific
system.

% Installation
\subsubsection{Installation}

We want to connect from \textit{machine$\_0$} to \textit{machine$\_1$} without having to type
the password every time.\bigbreak

\underline{\underline{Setup for machine$\_0$}}:

\begin{itemize}
    \item Create and ssh key; this key will have a private and public part.
    The public part will be copied to the receiving machine, i.e.,
    \textit{machine$\_1$}. T he private key will be kept in the sending machine,
    i.e., \textit{machine$\_0$}:
    \begin{minted}[frame=single]{bash}
ssh-keygen -t rsa -b 4096
    \end{minted}
    This will create a key with 4096 bits of encryption (suggested).
    \begin{itemize}
        \item To add a comment at the end of the key, use the 
        \mintinline{text}| -C| flag:
        \begin{minted}[frame=single]{bash}
ssh-keygen -t rsa -b 4096 -C "Comment here"
        \end{minted}
    \end{itemize}
    \item The command will ask for a location to save the key, make sure to
    choose the default location, which is \mintinline{text}|~/.ssh| directory;
    the name you can choose:
    \begin{minted}[frame=single]{text}
Enter file in which to save the key (/home/user/.ssh/id_rsa): /home/user/.ssh/name
    \end{minted}
    \item If you want to add a passphrase to the key, you can do so. This will
    add an extra layer of security to the key. If you do not want to add a
    passphrase, just press \keystroke{enter}:
    \begin{minted}[frame=single]{text}
Enter passphrase (empty for no passphrase):
    \end{minted}
    \item The key will be created and saved in the location you specified. There
    will be two files: \mintinline{text}|name| and \mintinline{text}|name.pub|,
    where \mintinline{text}|name| is the private key and
    \mintinline{text}|name.pub| is the public key.
    \item Copy the public key such that it can be added to the receiving
    machine.
    \item To make the connection automatic, you can use the file 
    \mintinline{text}|~/.ssh/config| to setup the connection. This file should
    look like:
    \begin{minted}[frame=single]{text}
Host machine_1
    HostName machine_1
    User user
    IdentityFile ~/.ssh/name
    \end{minted}
\end{itemize}

\underline{\underline{Setup for machine$\_1$}}:

\begin{itemize}
    \item With the public key in hand, copy the contents of the file into the
    \mintinline{text}|~/.ssh/authorized_keys| file in the receiving machine:
    \begin{minted}[frame=single]{bash}
cat name.pub >> ~/.ssh/authorized_keys
    \end{minted}
    \begin{itemize}
        \item For Windows, in Powershell (admin mode), use the following:
        \begin{minted}[frame=single]{bash}
Get-Content name.pub >> ~/.ssh/authorized_keys
        \end{minted}
    \end{itemize}
    \item Make sure that the permissions for the \mintinline{text}|~/.ssh|
    directory are set to \mintinline{text}|700| and the permissions for the
    \mintinline{text}|~/.ssh/authorized_keys| file are set to
    \mintinline{text}|600|:
    \begin{minted}[frame=single]{bash}
chmod 700 ~/.ssh
chmod 600 ~/.ssh/authorized_keys
    \end{minted}
    \begin{itemize}
        \item For Windows, there is no need to change the permissions.
    \end{itemize}
\end{itemize}