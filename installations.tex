%%%%%%%%%%%%%%%%%%%%%%%%%%%%%%%%%%%%%%%%%%%%%%%%%%%%%%%%%%%%%%%%%%%%%%%%%%%%%%%%
% "Installations and Setup"
%%%%%%%%%%%%%%%%%%%%%%%%%%%%%%%%%%%%%%%%%%%%%%%%%%%%%%%%%%%%%%%%%%%%%%%%%%%%%%%%
\section{\textquotedblleft Installations and Setup\textquotedblright\
Cheatsheet
}

%///////////////////////////////////////////////////////////////////////////////
% Linux - NeoVIM 
%///////////////////////////////////////////////////////////////////////////////

\subsection{Linux - NeoVIM}

% Installation
\subsubsection{Installation}

\begin{enumerate}
    %
    \item To install \textbf{NeoVIM} there are many methods. The suggested
    method is to use the \textit{snap} store, or equivalent, by using the
    command:
    \begin{minted}[frame=single]{bash}
$ sudo snap install nvim --classic
    \end{minted}
    This will install \textbf{NeoVIM} in the system.
    %
    \item The \textit{--classic} flag is used to install the software with
    full access to the system, which is required for some plugins and, most of
    the time, for the software to even install.
    %
    \item Make sure that the
    \textit{.bashrc} has the snap store path in the \mintinline{text}|PATH|
    variable at the start of the list:
    \begin{minted}[frame=single]{bash}
export PATH=/snap/bin:$PATH
    \end{minted}
    %
    \item To make sure the installation was successful, run the command:
    \begin{minted}[frame=single]{bash}
$ nvim --version
    \end{minted}
    Make sure that this is not a \textquotedblleft developer 
    version\textquotedblright; i.e., the latest stable version.
    %
\end{enumerate}

\subsubsection{Setup - Lazy: Package Manager}
This procedure is to perform the basic setup for Lazy and get \textbf{NeoVIM}
running in a basic way.
\begin{enumerate}
    \item Make sure that \textbf{NeoVIM} is properly installed.
    \item If the directory \mintinline{text}|~/.config/nvim| does not exist,
    create it:
    \begin{minted}[frame=single]{bash}
$ mkdir -p ~/.config/nvim
    \end{minted}
    %
    \item Create the file \mintinline{text}|~/.config/nvim/init.lua| file in the
    directory:
    \begin{minted}[frame=single]{bash}
touch ~/.config/nvim/init.lua
    \end{minted}
    This is done for the modern version of \textbf{NeoVIM}, which uses the
    \textit{lua} language for configuration.
    %
    \item Create the \mintinline{text}|~/.config/nvim/lua| directory, if 
    it does not exist:
    \begin{minted}[frame=single]{bash}
mkdir ~/.config/nvim/lua
    \end{minted}
    %
    \item In the \mintinline{text}|~/.config/nvim/lua| directory, create the
    three directories:
    \begin{itemize}
        \item config
        \item plugins
        \item util
    \end{itemize}
    This is where the different configuration files will be stored. Since 
    this is the default place where \textbf{NeoVim} will look for the different
    files and directories, it is a good idea to keep the default structure.
    %
    \item At this point, the file and directory structure should look like:
    \begin{minted}[frame=single]{bash}
<root == ~/.config/nvim>
| init.lua
|_ lua
    |_ config
    |_ plugins
    |_ util
    \end{minted}
    %
    \item To dowload the packages and give them the basic configuration, create
    an \mintinline{bash}|init.lua| in the \mintinline{bash}|config| directory: 
    \begin{minted}[frame=single]{bash}
<root == ~/.config/nvim>
| init.lua
|_ lua
    |_ config
    |   |_ init.lua    
    |_ plugins
    |_ util
    \end{minted}
    %
    \item Open the file \mintinline{bash}|<root>/lua/config/init.lua| file and
    add the following lines:
    \begin{minted}[frame=single]{lua}
local lazypath = vim.fn.stdpath("data") .. "/lazy/lazy.nvim"
if not vim.loop.fs_stat(lazypath) then
  vim.fn.system({
    "git",
    "clone",
    "--filter=blob:none",
    "https://github.com/folke/lazy.nvim.git",
    "--branch=stable", -- latest stable release
    lazypath,
  })
end
vim.opt.rtp:prepend(lazypath)

-- Example using a list of specs with the default options
-- Make sure to set `mapleader` before lazy so your mappings are correct
vim.g.mapleader = " " 

require("lazy").setup({
  "folke/which-key.nvim",
  { "folke/neoconf.nvim", cmd = "Neoconf" },
  "folke/neodev.nvim",
})
    \end{minted}
    %
    \item Use \textbf{NeoVIM} to open the file and, in \textbf{NeoVIM} command
    console mode (press \keystroke{esc}), run the command:
    command:
    \begin{minted}[frame=single]{text}
~
-------------------------------------
:checkhealth
    \end{minted}
\end{enumerate}

