%%%%%%%%%%%%%%%%%%%%%%%%%%%%%%%%%%%%%%%%%%%%%%%%%%%%%%%%%%%%%%%%%%%%%%%%%%%%%%%%
% "Make" Cheatsheet
%%%%%%%%%%%%%%%%%%%%%%%%%%%%%%%%%%%%%%%%%%%%%%%%%%%%%%%%%%%%%%%%%%%%%%%%%%%%%%%%
\section{\quoted{Make}\,Cheatsheet}

%///////////////////////////////////////////////////////////////////////////////
% General Structure
%///////////////////////////////////////////////////////////////////////////////
\subsection{General Structure}

The general structure of a \quoted{make} file is as follows:
\begin{mintedbox}{bash}
target: pre-requisites
    command
    command
    .
    .
    .
    command
\end{mintedbox}
\begin{itemize}
    \item target: is the name of the file generated by the \quoted{make} file.
    \item pre-requisites: are the files used to generate the target. These
    pre-requisites are also called\linebreak dependencies. Dependencies will be
    re-compiled if they have been modified since the last time the target was
    generated.
    \item command: is the command used to generate the target. There can be
    multiple commands per target.
\end{itemize}

%///////////////////////////////////////////////////////////////////////////////
% Using a "makefile".
%///////////////////////////////////////////////////////////////////////////////
\subsection{Using a \quoted{makefile}}
\begin{itemize}
    \item To run a simple \quoted{makefile} use the file structure shown below:
    \begin{mintedbox}{bash}
<root>
|___makefile
|___main.c
    \end{mintedbox}
    The content of the \quoted{makefile} is as follows:
    \begin{mintedbox}{bash}
all: main.c
    gcc -o main main.c
    \end{mintedbox}
    Use the file to compile the \quoted{main.c} file:
    \begin{mintedbox}{bash}
$ make
    \end{mintedbox}
    \item To include multiple commands add more targets:
    \begin{mintedbox}{bash}
default: main.c
    gcc -o main main.c

clean:
    rm main
    \end{mintedbox}
    When running a quoted{makefile} with multiple targets, the first target is the default target.
    To run a specific target use:
    \begin{mintedbox}{bash}
$ make clean
    \end{mintedbox}
    \newpage
    \item To include multiple pre-requisites add them to the target:
    \begin{mintedbox}{bash}
default: main.c main.h
    gcc -o main main.c
    \end{mintedbox}
\end{itemize}

%///////////////////////////////////////////////////////////////////////////////
% Using multiple dependencies.
%///////////////////////////////////////////////////////////////////////////////
\subsection{Using multiple dependencies}
\begin{itemize}
    \item The following \quoted{makefile} runs the three targets to build a program:
    \begin{mintedbox}{bash}
blah: blah.o
    gcc blah.o -o blah

blah.o: blah.c
    gcc -c blah.c -o blah.o

blah.c:
    echo "int main() { return 0; }" > blah.c
    \end{mintedbox}
    \subsubsection{How it works}
    \begin{enumerate}
        \item Make selects the target \quoted{blah}, because it is the first target in the
        \quoted{makefile}, i.e., the default target.
        \item \quoted{blah} depends on \quoted{blah.o}, so make looks for the
        target \quoted{blah.o}.
        \item \quoted{blah.o} depends on \quoted{blah.c}, so make looks for the target
        \quoted{blah.c}.
        \item \quoted{blah.c} is a phony target, so make runs the command associated with it.
        \item Make then goes back to \quoted{blah.o}, which is now available, and runs the command
        associated with it.
        \item Make then goes back to \quoted{blah}, which is now available, and runs the command
        associated with it.
    \end{enumerate}
    \item If \textbf{all} targets need to be run, use the \quoted{all} target:
    \begin{mintedbox}{bash}
all: one two three

one:
    echo "one"

two:
    echo "two"

three:
    echo "three"

clean:
    echo "clean"
    \end{mintedbox}
\end{itemize}

%///////////////////////////////////////////////////////////////////////////////
% Variables
%///////////////////////////////////////////////////////////////////////////////
\subsection{Variables}

Variables can only be strings.
\begin{itemize}
    \item To define a variable use $:=$ or $=$; the difference is that $:=$ is
    evaluated at the time of definition, while $=$ is evaluated at the time of
    use. For example:
    \begin{mintedbox}{bash}
x := $(shell date)
tt:
    echo $(x)
    sleep 2
    echo $(x)
    \end{mintedbox}
    will yield:
    \begin{mintedbox}{bash}
$ make tt
sex 22 jan 2021 14:56:08 -03
sex 22 jan 2021 14:56:08 -03
    \end{mintedbox}
    whereas:
    \begin{mintedbox}{bash}
x = $(shell date)
tt:
echo $(x)
sleep 2
echo $(x)
    \end{mintedbox}
    will yield:
    \begin{mintedbox}{bash}
$ make tt
sex 22 jan 2021 14:56:08 -03
sex 22 jan 2021 14:56:10 -03
    \end{mintedbox}
    Notice that the latter yields different results because the variable is
    evaluated at the time of use, not at the time of definition.
    \item Variables can be referenced using \quoted{\$()} or \quoted{\$\{\}}. For example:
    \begin{mintedbox}{bash}
x = 1
default:
    echo ${x}
    echo $(x)
    \end{mintedbox}
    \item The \quoted{\$@} character is a special character that represents the target. For example:
    \begin{mintedbox}{bash}
all: one two

one two:
    echo $@

# Is the same as:
# one:
#     echo one
# two:
#     echo two
    \end{mintedbox}
\end{itemize}

%///////////////////////////////////////////////////////////////////////////////
% Use of Wildcards.
%///////////////////////////////////////////////////////////////////////////////
\subsection{Use of Wildcards}

\begin{itemize}
    \item Wildcards can be used to select multiple files that match a pattern.
    \item There are three types of wildcards:
    \begin{enumerate}
        \item \quoted{*} matches any number of characters, including none.
        \item \quoted{\%} matches any single character, including none.
        \begin{itemize}
            \item This character can be used in \quoted{matching mode}, i.e., matches one or more
            characters in a string. This is called the \quoted{stem}.
            \item This character can be used in \quoted{replacing mode}, i.e., replaces the stem
            with the \quoted{replacement} string.
            \item Often used in rule definitions and some specific functions.
        \end{itemize}
    \end{enumerate}
\end{itemize}

%///////////////////////////////////////////////////////////////////////////////
% Implicit Rules
%///////////////////////////////////////////////////////////////////////////////
\subsection{Implict Rules}
There are some implicit rules when using \quoted{makefiles}:
\begin{itemize}
    \item Compiling a C program: \nminted{text}{n.o} is made automatically
    from \nminted{text}{n.c} with a command of the form:
    \begin{mintedbox}{bash}
$(CC) -c $(CPPFLAGS) $(CFLAGS) $^ -o $@
    \end{mintedbox}
    \item Compiling a C++ program: \nminted{text}{n.o} is made automatically
    from \nminted{text}{n.cc} or \nminted{text}{n.ccp} with a command of
    the form:
    \begin{mintedbox}{bash}
$(CXX) -c $(CPPFLAGS) $(CXXFLAGS) $^ -o $@
    \end{mintedbox}
    \item Linking a single object file: \nminted{text}{n} is made
    automatically from \nminted{text}{n.o} by running the command:
    \begin{mintedbox}{bash}
$(CC) $(LDFLAGS) $^ $(LOADLIBES) $(LDLIBS) -o $@
    \end{mintedbox}
    where:
    \begin{itemize}
        \item \nminted{text}{CC}: Program for compiling C programs; default
        is \nminted{text}{cc}.
        \item \nminted{text}{CXX}: Program for compiling C++ programs;
        default is \nminted{text}{g++}.
        \item \nminted{text}{CFLAGS}: Extra flags to give to the C compiler.
        \item \nminted{text}{CXXFLAGS}: Extra flags to give to the C++
        compiler.
        \item \nminted{text}{CXXFLAGS}: Extra flags to give to the C++
        preprocessor.
        \item \nminted{text}{LDFLAGS}: Extra flags to give to compilers when
        they are supposed to invoke the linker. For example,
        \nminted{text}{-L}.
    \end{itemize}
\end{itemize}

%///////////////////////////////////////////////////////////////////////////////
% Important Comments.
%///////////////////////////////////////////////////////////////////////////////
\subsection{Important Comments}
\begin{itemize}
    \item When indenting, use tabs, not spaces. Indenting in \quoted{make} is done with tabs, not
    spaces, i.e., tabs are mandatory.
\end{itemize}
