%%%%%%%%%%%%%%%%%%%%%%%%%%%%%%%%%%%%%%%%%%%%%%%%%%%%%%%%%%%%%%%%%%%%%%%%%%%%%%%%%%%%%%%%%%%%%%%%%%%%
%  "MySQL" Cheatsheet
%%%%%%%%%%%%%%%%%%%%%%%%%%%%%%%%%%%%%%%%%%%%%%%%%%%%%%%%%%%%%%%%%%%%%%%%%%%%%%%%%%%%%%%%%%%%%%%%%%%%
\section{\quoted{MySQL} Cheatsheet}

%///////////////////////////////////////////////////////////////////////////////////////////////////
%
%///////////////////////////////////////////////////////////////////////////////////////////////////

\subsection{Login to MySQL}

\begin{itemize}
    %
    \item Login to MySQL console:
    \begin{mintedbox}{bash}
mysql -u root -p
    \end{mintedbox}
    the password is in the notebook.
\end{itemize}

\subsection{Create a Database}

The following Python instructions create a database called \textit{mydb} in MySQL:
\begin{itemize}
    \item Create a pip virtual environment and activate it:
    \begin{mintedbox}{bash}
python3 -m venv venv
source venv/bin/activate
    \end{mintedbox}
    %
    \item Install the packages that will act as the connectors:
    \begin{mintedbox}{bash}
pip install mysql-connector mysql-connector-python mysql-connector-python-rf
    \end{mintedbox}
    %
    \item Using the following Python script, we can create a database called \textit{mydb}:
    \begin{mintedbox}{sql}
# Import the connector.
import mysql.connector as connector

# Create the database connection.
mydb = connector.connect(
    auth_plugin="mysql_native_password",
    host="localhost",
    password="<Root Password for the MySQL Server>",
    user="root",
)

# Run the commands.
my_cursor = mydb.cursor()
my_cursor.execute("CREATE DATABASE IF NOT EXISTS mydb")

# Print the existing databases.
my_cursor.execute("SHOW DATABASES")  # Show all the databases.

for db in my_cursor:
    # Names are encoded in bytes.
    print(tuple(x.decode("utf-8") for x in  db))
    \end{mintedbox}
    %
    \item Run the script and the output should be:
    \begin{mintedbox}{bash}
('information_schema',)
('mydb',)
('mysql',)
('performance_schema',)
('sys',)
    \end{mintedbox}
    %
    \item If the database appears in the list, then it has been created successfully.
\end{itemize}

\subsection{Integrate MySQL with Flask}

\begin{itemize}
    %
    \item Install the package:
    \begin{mintedbox}{bash}
sudo apt install libmysqlclient-dev
    \end{mintedbox}
    this will install the MySQL client library.
    \begin{mintedbox}{bash}
sudo apt install python3-dev
    \end{mintedbox}
    this will install the Python development headers.
    %
    \item Install the Python packages:
    \begin{mintedbox}{bash}
pip install cryptography Flask Flask-Login Flask-SocketIO Flask-SQLAlchemy mysqlclient pymysql
    \end{mintedbox}
    With these packages installed, we can now use MySQL with Flask.
\end{itemize}
